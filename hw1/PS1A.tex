\documentclass[12pt]{article}
\setlength{\oddsidemargin}{0in}
\setlength{\evensidemargin}{0in}
\setlength{\textwidth}{6.5in}
\setlength{\parindent}{0in}
\setlength{\parskip}{\baselineskip}
\usepackage{amsmath,amsfonts,amssymb}
\usepackage{graphicx}
\usepackage[]{algorithmicx}

\usepackage{fancyhdr}
\pagestyle{fancy}
\setlength{\headsep}{36pt}

\usepackage{hyperref}



\newcommand{\makenonemptybox}[2]{%
%\par\nobreak\vspace{\ht\strutbox}\noindent
\item[]
\fbox{% added -2\fboxrule to specified width to avoid overfull hboxes
% and removed the -2\fboxsep from height specification (image not updated)
% because in MWE 2cm is should be height of contents excluding sep and frame
\parbox[c][#1][t]{\dimexpr\linewidth-2\fboxsep-2\fboxrule}{
  \hrule width \hsize height 0pt
  #2
 }%
}%
\par\vspace{\ht\strutbox}
}
\makeatother

\begin{document}

\lhead{{\bf CSCI 3104, Algorithms \\ Problem Set 1a (10 points)} }
\rhead{Name: \fbox{Nick Bender} \\ ID: \fbox{106977096} \\ {\bf Profs.\ Hoenigman \& Agrawal\\ Fall 2019, CU-Boulder}}
\renewcommand{\headrulewidth}{0.5pt}

\phantom{Test}

\begin{small}
\textit{Advice 1}:\ For every problem in this class, you must justify your answer:\ show how you arrived at it and why it is correct. If there are assumptions you need to make along the way, state those clearly.
\vspace{-3mm} 

\textit{Advice 2}:\ Verbal reasoning is typically insufficient for full credit. Instead, write a logical argument, in the style of a mathematical proof.\\
\vspace{-3mm} 

\textbf{Instructions for submitting your solution}:
\vspace{-5mm} 

\begin{itemize}
	\item The solutions \textbf{should be typed} and we cannot accept hand-written solutions. \href{http://ece.uprm.edu/~caceros/latex/introduction.pdf}{Here's a short intro to Latex.}
	\item You should submit your work through \href{https://www.gradescope.com/courses/59294}{\textbf{Gradescope}} only.
	\item If you don't have an account on it, sign up for one using your CU email. You should have gotten an email to sign up. If your name based CU email doesn't work, try the identikey@colorado.edu version. 
	\item Gradescope will only accept \textbf{.pdf} files (except for code files that should be submitted separately on Gradescope if a problem set has them) and \textbf{try to fit your work in the box provided}. 
	\item You cannot submit a pdf which has less pages than what we provided you as Gradescope won't allow it. 
\end{itemize}
\vspace{-4mm} 
\end{small}

\hrulefill
\pagebreak

\begin{enumerate}

	\item	{\itshape (3 pts) What are the three components of a loop invariant proof? Write a one-sentence description for each one.}
\makenonemptybox{3in}{
1. initialization: loop inv true prior to first iteration of the for loop\\
2. maintenance: loop inv is true immediately before or after the loop body executes\\
3. termination: loop inv is true after the loop terminates
}

\item {\itshape (6 pts total) Identify the loop invariant in the following algorithms.}
\begin{enumerate}
\item \label{2a} 
	\begin{small}
	\begin{verbatim}
	FindMaxElement(A) : //suppose array A is not empty
	    ret = A[0]
	    for i = 1 to length(A)-1 {
	        if A[i] > ret{
	            ret = A[i]	           
	    }}
	    return ret
	\end{verbatim}
	\end{small}

\makenonemptybox{2in}{

init: i = 1 \\
      ret = A[0]\\
main: i increments by 1 \\
      ret changes with if condition\\
term: i = length(A)-1 always (the loop doesn't terminate early for any reason) \\
      ret = max value of A
      
}
\pagebreak

\item \label{2b} 
	\begin{small}
	\begin{verbatim}
	FindElement(A, n) : //suppose no duplicates in array A and array A is not empty
	    ret = -1 //index -1 implies the element haven't been found yet
	    for i = 0 to length(A)-1 {
	        if A[i] == n{
	            ret = i	           
	    }}
	    return ret
	\end{verbatim}
	\end{small}
% Your answer goes here. Here's an example where we didn't bother with a box because the space just goes to the end of the page. Same comment as before about extra space applies.
\makenonemptybox{2.5in}{
init: i = 0 \\
      ret = -1\\
main: i increments by 1 \\
      ret might change with if condition\\
term: i = length(A)-1 always (the loop doesn't terminate early for any reason) \\
      
}
\pagebreak

\item \label{2c}
    \begin{small}
	\begin{verbatim}
	SumArray(A) : //suppose array A is not empty
	    sum = 0
	    for i = 0 to length(A)-1 {
	        sum += A[i]    
	    }
	    return sum
	\end{verbatim}
	\end{small}
\makenonemptybox{2.5in}{
init: i = 0 \\
      sum = 0\\
main: i increments by 1 \\
term: i = length(A)-1 always (the loop doesn't terminate early for any reason) \\
}
\end{enumerate}

\pagebreak
	\item {\itshape (1 pt)  If r is a real number not equal to 1, then for every n $\geq$ 0, 
	\[
	\sum_{i=0}^{n} r^{i} = \frac{ (1 - r^{n-1})}{ (1 - r)}.
	\]
	
	\noindent Provide the first two steps of a proof by induction i.e. base case and the inductive hypothesis. You will be asked to complete this proof later in \textbf{PS1b}.}

	\makenonemptybox{3in}{
	base case: \\
	consider the case n=0 and we let r=2\\
	$\sum_{i=0}^{0}2^{i} = \frac{ (1 - 2^{0-1} ) }{ (1 - 2) }$\\
	1 = $\frac{ 0.5}{ -1 }$\\
	Houston we have a problem\\
	consider the case n=1 and we let r=2\\
	$\sum_{i=0}^{1}2^{i} = \frac{ (1 - 2^{1-1} ) }{ (1 - 2) }$\\
	1+2 = $\frac{ 0}{ -1 }$\\
	Houston we have an even bigger problem problem\\
	\\	
	Inductive hypothesis: \\
	fix k $\in$ $R$ and k $\neq$ 1 and suppose n $\geq$ 0\\
	$\sum_{i=0}^{0} 2^{i} = \frac{ (1 - r^{0-1})}{ (1 - r)}$\\
	}
	\pagebreak


	
\end{enumerate}


\end{document}


